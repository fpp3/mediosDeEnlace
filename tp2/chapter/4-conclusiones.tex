\chapter{Conclusiones}

En este trabajo práctico realizamos mediciones de la Relación de Onda Estacionaria (ROE) en diferentes elementos radiantes utilizando un vatímetro direccional Bird. El objetivo fue analizar cómo la longitud de cada elemento afecta la adaptación de impedancias entre la línea de transmisión y la antena.

Los resultados experimentales mostraron que el elemento de 10 cm presentó el mejor desempeño ya que posee el menor valor de ROE. Esto significa que este posee la mejor adaptación, permitiendo que la mayor parte de la potencia transmitida llegue efectivamente a la antena en lugar de ser reflejada de vuelta hacia el transmisor.

Durante las mediciones observamos que a medida que nos alejamos de la longitud óptima del elemento radiante, el ROE aumenta significativamente. Este comportamiento tiene sentido desde el punto de vista teórico: la impedancia de entrada de un dipolo varía con su longitud eléctrica, y la máxima eficiencia se alcanza cuando las dimensiones son apropiadas para la frecuencia de trabajo.

El uso del vatímetro Bird nos permitió medir separadamente la potencia incidente en la antena y la que se refleja, facilitando el cálculo directo del ROE.

En conclusión, este trabajo nos permitió comprobar en la práctica conceptos que habíamos visto en teoría sobre adaptación de impedancias y ondas estacionarias. Quedó demostrado que elegir correctamente las dimensiones del elemento radiante es fundamental para lograr una transmisión eficiente, y que el ROE es una herramienta muy útil para evaluar qué tan bien está funcionando nuestro sistema de antena; también, un aspecto importante que aprendimos es que un ROE elevado no solo representa pérdida de potencia si no que también puede causar problemas serios en el equipo ya que las ondas estacionarias generan picos de voltaje y corriente en ciertos puntos de la línea que pueden dañar el transmisor.

