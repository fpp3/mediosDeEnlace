\chapter{Introduccion}

En el presente trabajo práctico se aborda la medición del ROE (Relación de Onda Estacionaria) en sistemas de
radiofrecuencia, un parámetro fundamental que permite evaluar la adaptación de impedancias entre el transmisor y la
antena.

El objetivo principal es comprender de manera práctica cómo medir la potencia incidente y reflejada en una línea de
transmisión utilizando un vatímetro de RF, y a partir de estos valores calcular el ROE del sistema. Esta relación es
crucial ya que un valor elevado de ROE indica desadaptación, lo que puede provocar pérdidas de potencia, reducción en la
eficiencia del sistema y potenciales daños en el equipamiento transmisor.

Para el desarrollo de esta práctica se utilizará un transmisor UHF Alinco DR 430, un vatímetro Bird modelo 43, y
diferentes irradiantes de longitudes variadas montados sobre una base magnética. El análisis comparativo de los
diferentes irradiantes permitirá determinar cuál presenta la mejor adaptación para la frecuencia de trabajo de 450 MHz.

