\chapter{Laboratorio}
  Para la realización de la actividad practica de laboratorio, se requieren los siguientes materiales/elementos:
  \begin{itemize}
    \item Generador de onda cuadrada ($f = 100KHz$, $V = 3Vpp$),
    \item Osciloscopio de doble trazo,
    \item Multimetro,
    \item Cable coaxial $75 \Omega$,
    \item Kit de medición de adaptación en lineas de transmisión.
  \end{itemize}

  Todos los elementos son provistos por el laboratorio de comunicaciones de la UTN FRC.

  El circuito propuesto se puede ver en la figura \ref{fig:crkt}

  \begin{figure}[!ht]
    \centering
    \resizebox{.9\textwidth}{!}{%
\begin{circuitikz}
\tikzstyle{every node}=[font=\LARGE]
\node [font=\LARGE] at (5,13.25) {Generador};
\draw  (3,15.25) rectangle (7,11.25);
\node [font=\LARGE] at (22.5,13.25) {$Z_L$};
\draw  (20.5,15.25) rectangle (24.5,11.25);
\draw  (10,13.25) circle (0.75cm);
\draw  (17.5,13.25) circle (0.75cm);
\draw [short] (7,13.25) -- (9.25,13.25);
\draw [short] (10.75,13.25) -- (16.75,13.25);
\draw [short] (18.25,13.25) -- (20.5,13.25);
\node [font=\LARGE] at (13.75,10.5) {Osciloscopio};
\node [font=\LARGE] at (13.75,9.5) {\textbf{CH1    CH2}};
\draw  (11.75,11.5) rectangle (15.75,9.25);
\draw [short] (8,13.25) -- (8,13.25);
\draw [short] (8,13.25) -- (8,8.25);
\draw [short] (8,8.25) -- (12.75,8.25);
\draw [short] (12.75,8.25) -- (12.75,9.25);
\draw [short] (14.75,9.25) -- (14.75,8.25);
\draw [short] (14.75,8.25) -- (19.5,8.25);
\draw [short] (19.5,8.25) -- (19.5,13.25);
\draw [short] (10,12.5) -- (10,11.5);
\draw [short] (9.5,11.5) -- (10.5,11.5);
\draw [short] (17.5,12.5) -- (17.5,11.5);
\draw [short] (17,11.5) -- (18,11.5);
\node [font=\LARGE] at (13.75,13.75) {Linea de transmision};
\node [font=\LARGE] at (10,14.5) {\textbf{A}};
\node [font=\LARGE] at (17.5,14.5) {\textbf{B}};
\end{circuitikz}
}%

    \caption{circuito a implementar.}
    \label{fig:crkt}
  \end{figure}
  
  \section{Actividades}
    \begin{enumerate}
      \item En base al circuito propuesto, ubicar con la llave selectora de impedancia del kit, una resistencia de carga
        $Z_L = 75 \ \Omega$ en el punto B. Observar las formas de onda en los puntos A y B (la línea de transmisión está
        adaptada: $Z_L = Z_O = 75 \ \Omega$).
        \begin{gather*}
          \text{Time/Div: } 2 \mu s \\
          \text{\textbf{CH1: }} 1\sfrac{V}{div} \quad V_{pp} = 2v \\
          \text{\textbf{CH2: }} 1\sfrac{V}{div} \quad V_{pp} = 2v
        \end{gather*}
        \begin{figure}[H]
          \centering
          \includegraphics[width=.8\textwidth]{pictures/stup.jpeg}
          \caption{kit de medición de adaptación en lineas de transmisión usado.}
        \end{figure}
      \item Graficar la forma de onda de los canales Ch1 y Ch2.
        \begin{figure}[!ht]
          \centering
          \includegraphics[width=.8\textwidth]{pictures/osc_lin_orig.jpeg}
          \caption{señales capturadas en el osciloscopio con la linea adaptada.}
          \label{fig:sig_plot1}
        \end{figure}
      \item Interpretar lo obtenido.

        En la figura \ref{fig:sig_plot1} se pueden apreciar los dos puntos de medición en el osciloscopio, siendo la
        señal superior la medida en el punto A y la inferior en el punto B. A simple vista no se aprecia ninguna
        alteración en la linea. Esto se debe a que la linea en si es relativamente corta, por lo que retardos de
        propagación no son apreciables, y además la linea se encuentra prácticamente adaptada, con una carga de $Z_L =
        75 \Omega$, que es la misma que la impedancia nominal de la linea de transmisión empleada.
      \item En el terminal B de la línea de transmisión, ubicar con la llave selectora de impedancia del kit, el
        potenciómetro ($Z_L$ = variable). Variar el potenciómetro desde cero al máximo. Encontrar la posición en que
        desaparecen las distorsiones, observando la señal a la salida del generador y en la carga.
        \begin{figure}[!ht]
          \centering
          \includegraphics[width=.8\textwidth]{pictures/osc_lin_adap.jpg}
          \caption{señal capturada sin distorsión para un $Z_L = 72 \Omega$. Escala de tiempo $\sfrac{1 \mu s}{div}$}
          \label{fig:sig_plot2}
        \end{figure}
        Mantener la posición obtenida por el potenciómetro. Cambiar el selector a la posición carga $Z_L = 75 \ \Omega$.
        Medir el valor de la resistencia $R_{POT}$ del potenciómetro con un óhmetro.
        \begin{equation*}
          R_{POT} = 72\Omega
        \end{equation*}
      \item Interpretar cualitativamente lo obtenido.

        Habiendo reajustado la impedancia de la carga, y haciendo un aumento en la escala temporal y de voltaje, podemos
        apreciar que la reflexión en la linea es prácticamente nula. Observando mas detenidamente la señal ubicada en la
        parte inferior del osciloscopio, que seria el voltaje presente en la carga $V_L$, podemos ver que cerca de
        llegar a estabilizarse, la señal presenta una pequeña curva. Esto puede ser porque la linea puede tener cierta
        componente reactiva, la cual no es posible eliminar usando una carga puramente resistiva. Esto produce un muy
        pequeño pero apreciable índice de reflexión, lo que causa esa curva antes de que la señal se estabilice.
      \item Con la línea adaptada ($Z_L = Z_O = 75 \ \Omega$), medir el tiempo de retardo ($T_r$) entre los puntos A y B.
        \begin{figure}[!ht]
          \centering
          \includegraphics[width=.8\textwidth]{pictures/osc_prop_del.jpeg}
          \caption{señal capturada midiendo el tiempo de retardo de propagación. Escala en $\sfrac{50ns}{div}$.}
          \label{fig:sig_plot3}
        \end{figure}
        \begin{equation*}
          T_r = 350ns
        \end{equation*}
      \item Con la línea adaptada ($Z_L = Z_O = 75 \ \Omega$), calcular la atenuación de la línea de transmisión
        ($A_{tt}$) entre los puntos A y B, medida en decibles (dB).
        \begin{figure}[!ht]
          \centering
          \begin{minipage}{0.45\textwidth}
            \centering
            \begin{align*}
              E_A = 2V_{pp} \\
              E_B = 2V_{pp}
            \end{align*}
          \end{minipage}
          \hfill
          \begin{minipage}{0.45\textwidth}
            \centering
            \begin{equation*}
              A_{tt} = 20 \log \left( \frac{E_B}{E_A} \right) = 0dB
            \end{equation*}
          \end{minipage}
        \end{figure}
        \begin{figure}[!ht]
          \centering
          \includegraphics[width=.8\textwidth]{pictures/osc_att.jpg}
          \caption{señal capturada midiendo la diferencia de voltaje en los puntos A y B.}
        \end{figure}
      \item Calcular el factor de propagación ($F_p$) de la línea de transmisión por intermedio de la velocidad de
        propagación $V_p$.
        \begin{gather*}
          V_p = \frac{L_{lt}}{T_r} = \frac{5.33m}{350ns} = 15.23 \times 10^6 \sfrac{m}{s} \\
          F_p = \frac{V_p}{c} = 0.05
        \end{gather*}
        Conocido el valor del factor de propagación Fp, es posible calcular el valor de la permitividad relativa
        \text{$\epsilon_r$} del cable coaxial
        \begin{equation*}
          \epsilon_r = \frac{1}{F_p^2} = 388.1
        \end{equation*}
      \item En el punto B de la línea de transmisión, ubicar la llave selectora de impedancia del kit en circuito
        abierto $Z_L = \infty$. Observar las formas de onda en los puntos A y B.
      \item Graficar la forma de onda de los canales Ch1 y Ch2.
        \begin{figure}[!ht]
          \centering
          \includegraphics[width=.8\textwidth]{pictures/osc_lin_ca.jpeg}
          \caption{señales capturadas por el osciloscopio con la linea en corto.}
          \label{fig:sig_plot4}
        \end{figure}
      \item Interpretar lo obtenido.

        Al permanecer la linea en CA, no hubo cambios significativos de la señal de salida. Teóricamente, deberíamos
        haber apreciado como la onda se reflejaba a si misma, cancelando el potencial del generador, y gradualmente
        aumentando la tensión en la salida. Esto se debe a que cuando $Z_L = \infty$, si reemplazamos en la
        ecuación \ref{eq:coef.refl} tendríamos:
        \begin{align*}
          \Gamma_R &= \frac{\infty - Z_0}{\infty + Z_0} \\
          \Gamma_R &= \lim_{Z_L \to \infty} \frac{Z_L - Z_0}{Z_L + Z_0} \\
          \Gamma_R &= 1
        \end{align*}

        Posiblemente no hayamos podido apreciar el efecto por la corta longitud de la onda y la resolución del
        osciloscopio. También pudo haber sido problema del kit, ya que este nos genero bastantes problemas al momento de
        manipularlo, generando falsos contactos y ruido indeseado.
      \item En el punto B de la línea de transmisión, ubicar la llave selectora de impedancia del kit en cortocircuito
        ($Z_L = 0$).  Observar las formas de onda en los puntos A y B.
      \item Graficar la forma de onda de los canales Ch1 y Ch2.
        \begin{figure}[!ht]
          \centering
          \includegraphics[width=.8\textwidth]{pictures/osc_lin_cc.jpg}
          \caption{señales capturadas por el osciloscopio con la linea abierta.}
          \label{fig:sig_plot5}
        \end{figure}
      \item Interpretar lo obtenido.

        Al permanecer la linea en CC, podemos apreciar que el potencial en la carga es prácticamente nulo durante los
        $360^\circ$ de la señal inyectada, solo presentando un pequeño pulso cuando se presentaba el flanco positivo del
        generador, en ambos puntos A y B. Esto se debe a que toda la señal que transmite el generador, es reflejada con
        la fase invertida, lo que provoca que se opongan entre si y anulen la totalidad del efecto. Considerando que 
        $Z_L = 0$, si reemplazamos en la ecuación \ref{eq:coef.refl} tendríamos:
        \begin{align*}
          \Gamma_R &= \frac{0 - Z_0}{0 + Z_0} \\
          \Gamma_R &= -1
        \end{align*}

        Si bien, no deberíamos haber visto nada en el osciloscopio, el pequeño pulso puede ser que se presente debido a
        la velocidad de propagación de la onda, la cual es bastante baja como vimos en los cálculos previos.
    \end{enumerate}
