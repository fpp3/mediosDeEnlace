\chapter{Laboratorio}
  Para la realización de la actividad practica de laboratorio, se requieren los siguientes materiales/elementos:
  \begin{itemize}
    \item Generador de onda cuadrada ($f = 100KHz$, $V = 3Vpp$),
    \item Osciloscopio de doble trazo,
    \item Multimetro,
    \item Cable coaxial $50 \Omega$,
    \item Kit de medición de adaptación en lineas de transmisión.
  \end{itemize}

  Todos los elementos son provistos por el laboratorio de comunicaciones de la UTN FRC.

  El circuito propuesto se puede ver en la figura \ref{fig:crkt}

  \begin{figure}[!ht]
    \centering
    \resizebox{.9\textwidth}{!}{%
\begin{circuitikz}
\tikzstyle{every node}=[font=\LARGE]
\node [font=\LARGE] at (5,13.25) {Generador};
\draw  (3,15.25) rectangle (7,11.25);
\node [font=\LARGE] at (22.5,13.25) {$Z_L$};
\draw  (20.5,15.25) rectangle (24.5,11.25);
\draw  (10,13.25) circle (0.75cm);
\draw  (17.5,13.25) circle (0.75cm);
\draw [short] (7,13.25) -- (9.25,13.25);
\draw [short] (10.75,13.25) -- (16.75,13.25);
\draw [short] (18.25,13.25) -- (20.5,13.25);
\node [font=\LARGE] at (13.75,10.5) {Osciloscopio};
\node [font=\LARGE] at (13.75,9.5) {\textbf{CH1    CH2}};
\draw  (11.75,11.5) rectangle (15.75,9.25);
\draw [short] (8,13.25) -- (8,13.25);
\draw [short] (8,13.25) -- (8,8.25);
\draw [short] (8,8.25) -- (12.75,8.25);
\draw [short] (12.75,8.25) -- (12.75,9.25);
\draw [short] (14.75,9.25) -- (14.75,8.25);
\draw [short] (14.75,8.25) -- (19.5,8.25);
\draw [short] (19.5,8.25) -- (19.5,13.25);
\draw [short] (10,12.5) -- (10,11.5);
\draw [short] (9.5,11.5) -- (10.5,11.5);
\draw [short] (17.5,12.5) -- (17.5,11.5);
\draw [short] (17,11.5) -- (18,11.5);
\node [font=\LARGE] at (13.75,13.75) {Linea de transmision};
\node [font=\LARGE] at (10,14.5) {\textbf{A}};
\node [font=\LARGE] at (17.5,14.5) {\textbf{B}};
\end{circuitikz}
}%

    \caption{circuito a implementar.}
    \label{fig:crkt}
  \end{figure}
  
  \section{Actividades}
    \begin{enumerate}
      \item En base al circuito propuesto, ubicar con la llave selectora de impedancia del kit, una resistencia de carga
        $Z_L = 75 \ \Omega$ en el punto B. Observar las formas de onda en los puntos A y B (la línea de transmisión está
        adaptada: $Z_L = Z_O = 75 \ \Omega$).
        \begin{gather*}
          \text{Time/Div:} *s \\
          \text{\textbf{CH1:}} *\sfrac{V}{div} \quad V_{pp} = *v \\
          \text{\textbf{CH2:}} *\sfrac{V}{div} \quad V_{pp} = *v
        \end{gather*}
      \item Graficar la forma de onda de los canales Ch1 y Ch2.
        \begin{figure}[!ht]
          \centering
          %señales
          \caption{señales capturadas en el osciloscopio con la linea adaptada.}
          \label{fig:sig_plot1}
        \end{figure}
      \item Interpretar lo obtenido.
      \item En el terminal B de la línea de transmisión, ubicar con la llave selectora de impedancia del kit, el
        potenciómetro ($Z_L$ = variable). Variar el potenciómetro desde cero al máximo. Encontrar la posición en que
        desaparecen las distorsiones, observando la señal a la salida del generador y en la carga.
        \begin{figure}[!ht]
          \centering
          %señales
          \caption{señal capturada sin distorsión para un $Z_L = *$.}
          \label{fig:sig_plot2}
        \end{figure}
        Mantener la posición obtenida por el potenciómetro. Cambiar el selector a la posición carga $Z_L = 75 \ \Omega$.
        Medir el valor de la resistencia $R_{POT}$ del potenciómetro con un óhmetro.
        \begin{equation*}
          R_{POT} = *\Omega
        \end{equation*}
      \item Interpretar cualitativamente lo obtenido.
      \item Con la línea adaptada ($Z_L = Z_O = 75 \ \Omega$), medir el tiempo de retardo ($T_r$) entre los puntos A y B.
        \begin{figure}[!ht]
          \centering
          %señales
          \caption{señal capturada midiendo el tiempo de retardo de propagación. Escala en $\sfrac{s}{div}$.}
          \label{fig:sig_plot3}
        \end{figure}
        \begin{equation*}
          T_r = *s
        \end{equation*}
      \item Con la línea adaptada ($Z_L = Z_O = 75 \ \Omega$), calcular la atenuación de la línea de transmisión
        ($A_{tt}$) entre los puntos A y B, medida en decibles (dB).
        \begin{figure}[!ht]
          \centering
          \begin{minipage}{0.45\textwidth}
            \centering
            \begin{align*}
              E_A = *V_{pp} \\
              E_B = *V_{pp}
            \end{align*}
          \end{minipage}
          \hfill
          \begin{minipage}{0.45\textwidth}
            \centering
            \begin{equation*}
              A_{tt} = 20 \log \left( \frac{E_B}{E_A} \right) = *dB
            \end{equation*}
          \end{minipage}
        \end{figure}
      \item Calcular el factor de propagación ($F_p$) de la línea de transmisión por intermedio de la velocidad de
        propagación $V_p$.
        \begin{gather*}
          V_p = \frac{L_{lt}}{T_r} = * \sfrac{m}{s} \\
          F_p = \frac{V_p}{c} = *
        \end{gather*}
        Conocido el valor del factor de propagación Fp, es posible calcular el valor de la permitividad relativa
        \text{$\epsilon_r$} del cable coaxial
        \begin{equation*}
          \epsilon_r = \frac{1}{F_p^2} = * 
        \end{equation*}
      \item En el punto B de la línea de transmisión, ubicar la llave selectora de impedancia del kit en circuito
        abierto $Z_L = \infty$. Observar las formas de onda en los puntos A y B.
      \item Graficar la forma de onda de los canales Ch1 y Ch2.
        \begin{figure}[!ht]
          \centering
          %señales
          \caption{señales capturadas por el osciloscopio.}
          \label{fig:sig_plot4}
        \end{figure}
      \item Interpretar lo obtenido.
      \item En el punto B de la línea de transmisión, ubicar la llave selectora de impedancia del kit en cortocircuito
        (ZL = 0).  Observar las formas de onda en los puntos A y B.
      \item Graficar la forma de onda de los canales Ch1 y Ch2.
        \begin{figure}[!ht]
          \centering
          %señales
          \caption{señales capturadas por el osciloscopio.}
          \label{fig:sig_plot5}
        \end{figure}
      \item Interpretar lo obtenido.
    \end{enumerate}
