\chapter{Laboratorio}
  Para la realizacion de la actividad practica de laboratorio, se requieren los siguientes materiales/elementos:
  \begin{itemize}
    \item Generador de onda cuadrada ($f = 100KHz$, $V = 3Vpp$),
    \item Osciloscopio de doble trazo,
    \item Multimetro,
    \item Cable coazial $50 \Omega$,
    \item Kit de medicion de adaptacion en lineas de transmision.
  \end{itemize}

  Todos los elementos son provistos por el laboratorio de comunicaciones de la UTN FRC.

  El circuito propuesto se puede ver en la figura \ref{fig:crkt}

  \begin{figure}[!ht]
    \centering
    \resizebox{.9\textwidth}{!}{%
\begin{circuitikz}
\tikzstyle{every node}=[font=\LARGE]
\node [font=\LARGE] at (5,13.25) {Generador};
\draw  (3,15.25) rectangle (7,11.25);
\node [font=\LARGE] at (22.5,13.25) {$Z_L$};
\draw  (20.5,15.25) rectangle (24.5,11.25);
\draw  (10,13.25) circle (0.75cm);
\draw  (17.5,13.25) circle (0.75cm);
\draw [short] (7,13.25) -- (9.25,13.25);
\draw [short] (10.75,13.25) -- (16.75,13.25);
\draw [short] (18.25,13.25) -- (20.5,13.25);
\node [font=\LARGE] at (13.75,10.5) {Osciloscopio};
\node [font=\LARGE] at (13.75,9.5) {\textbf{CH1    CH2}};
\draw  (11.75,11.5) rectangle (15.75,9.25);
\draw [short] (8,13.25) -- (8,13.25);
\draw [short] (8,13.25) -- (8,8.25);
\draw [short] (8,8.25) -- (12.75,8.25);
\draw [short] (12.75,8.25) -- (12.75,9.25);
\draw [short] (14.75,9.25) -- (14.75,8.25);
\draw [short] (14.75,8.25) -- (19.5,8.25);
\draw [short] (19.5,8.25) -- (19.5,13.25);
\draw [short] (10,12.5) -- (10,11.5);
\draw [short] (9.5,11.5) -- (10.5,11.5);
\draw [short] (17.5,12.5) -- (17.5,11.5);
\draw [short] (17,11.5) -- (18,11.5);
\node [font=\LARGE] at (13.75,13.75) {Linea de transmision};
\node [font=\LARGE] at (10,14.5) {\textbf{A}};
\node [font=\LARGE] at (17.5,14.5) {\textbf{B}};
\end{circuitikz}
}%

    \caption{circuito a implementar.}
    \label{fig:crkt}
  \end{figure}
  
  \section{Actividades}
    En base al circuito propuesto, ubicamos con la llave selectora de impedancia del kit, una resistencia de carga
    $Z_L = 75 \Omega$ en el punto \textbf{B}. Observando las formas de onda en los puntos \textbf{A} y \textbf{B},
    considerando que la linea de transmision esta adaptada, obtuvimos:
    \begin{gather*}
      \text{Time/Div:} *s \\
      \text{\textbf{CH1:}} *\sfrac{V}{div} \quad V_{pp} = *v \\
      \text{\textbf{CH2:}} *\sfrac{V}{div} \quad V_{pp} = *v
    \end{gather*}

    En la figura \ref{fig:sig_plot1} puede observar las señales capturadas del osciloscopio.
    \begin{figure}[!ht]
      \centering
      %señales
      \caption{señales capturadas en el osciloscopio con la linea adaptada.}
      \label{fig:sig_plot1}
    \end{figure}
