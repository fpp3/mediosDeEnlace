\chapter{Introduccion}
El estudio de la propagación de señales en líneas de transmisión es fundamental para comprender el comportamiento de los
sistemas eléctricos y de comunicaciones cuando las dimensiones físicas del medio no pueden considerarse despreciables
respecto a la longitud de onda de la señal. En este trabajo se analiza la respuesta de una línea coaxial ante la
aplicación de una señal escalón, observando los fenómenos asociados al régimen transitorio y estacionario.

A través de la experimentación, se busca determinar parámetros característicos como la impedancia característica $Z_0$,
el coeficiente de reflexión, el tiempo de retardo, la velocidad y el factor de propagación, así como las pérdidas por
atenuación. Además, se estudia la influencia de la impedancia de carga sobre la forma de onda observada y el grado de
adaptación entre los extremos de la línea. Este análisis permite relacionar los conceptos teóricos con las observaciones
prácticas, reforzando la comprensión del comportamiento electromagnético de las líneas de transmisión.
