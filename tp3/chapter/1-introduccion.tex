\chapter{Introducción}

En el presente trabajo práctico se aborda la adaptación de líneas de transmisión mediante el uso de un \emph{stub simple}, una técnica ampliamente utilizada en sistemas de radiofrecuencia para corregir desadaptaciones de impedancia.  
El objetivo principal es analizar de manera práctica cómo la conexión de un \emph{stub} permite minimizar la potencia reflejada en una línea, logrando una transferencia de energía más eficiente entre el generador y la carga.

Cuando la impedancia de la carga no coincide con la impedancia característica de la línea, parte de la onda incidente se refleja hacia el generador, originando una onda estacionaria.  
Este fenómeno, además de provocar pérdidas de potencia, puede generar sobrecalentamiento y deterioro en los equipos de transmisión.  
Por ello, comprender el comportamiento de la reflexión y la forma de compensarla resulta esencial en el diseño y mantenimiento de sistemas de comunicación.

Para el desarrollo de la práctica se utiliza un transmisor UHF \emph{Alinco DR-430}, un vatímetro \emph{Bird 43}, una línea de transmisión de $50~\Omega$ y una carga de prueba.  
A partir de las mediciones de potencia incidente y reflejada, se calcula la Relación de Onda Estacionaria (ROE) y el coeficiente de reflexión, comparando los resultados obtenidos antes y después de la conexión del \emph{stub}.  
El análisis final permite verificar experimentalmente la mejora en la adaptación y la disminución de la potencia reflejada cuando la línea se encuentra correctamente ajustada.
