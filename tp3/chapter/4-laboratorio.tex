\chapter{Laboratorio}
  Para la realización de la actividad practica de laboratorio, se requieren los siguientes materiales/elementos:
  \begin{itemize}
    \item Generador de UHF (transceptor ALINCO UHF),
    \item Vatímetro de UHF,
    \item Fuente de alimentación $12V$,
    \item Línea de transmisión $Z_O = 50 \Omega$,
    \item Impedancias de carga de prueba $Z_{L1}$,
    \item Cables varios (numerados 10, 11, 12) para interconexión (ya instalados),
    \item Conectores BNC hembra-hembra, BNC Triple hembra (T) y adaptador UHF hembra a N macho.
  \end{itemize}

  \section{Actividades}
    \begin{enumerate}
      \item Conectar el generador a la frecuencia de trabajo $f = 410 MHz$.
      \item Con el vatímetro, verificar la potencia entregada por el transmisor (debe indicar LOW).
      \item Conectar la impedancia de carga $Z_L$ al vatímetro.
      \item Medir la potencia incidente $P_i$  y la potencia reflejada $P_r$.
        \begin{equation*}
          P_i = *W \quad P_r = *W
        \end{equation*}
      \item Calcular el coeficiente de reflexión $\Gamma$ y la Relación de Onda Estacionaria ROE sin ningún stub (ss) conectado a la línea de transmisión.
        \begin{equation*}
          \Gamma_{ss} = * \quad ROE_{ss} = *
        \end{equation*}
      \item Comprobar en base a lo calculado en el ejercicio que el Stub y la distancia al Stub sean iguales.
      \item Conectar el adaptador BNC tipo T luego del vatímetro.
      \item Conectar la carga mediante el segmento de línea calculado y en el extremo libre, conectar el stub adecuado.
      \item Medir la potencia incidente $P_i$  y la potencia reflejada $P_r$.
        \begin{equation*}
          P_i = *W \quad P_r = *W
        \end{equation*}
      \item Calcular el coeficiente de reflexión $\Gamma$ y la Relación de Onda Estacionaria ROE con el stub (cs) conectado a la línea de transmisión.
        \begin{equation*}
          \Gamma_{cs} = * \quad ROE_{cs} = *
        \end{equation*}
    \end{enumerate}

  \section{Cuestionario}
    \begin{enumerate}
      \item ¿Qué implica que la línea de transmisión esté adaptada?
      \item ¿Qué conclusión de obtiene con las mediciones de solamente la carga y después con la introducción de un Stub?
      \item ¿Qué función tiene la carga utilizada?
    \end{enumerate}
