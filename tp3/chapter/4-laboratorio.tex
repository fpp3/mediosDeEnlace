\chapter{Laboratorio}
  Para la realización de la actividad practica de laboratorio, se requieren los siguientes materiales/elementos:
  \begin{itemize}
    \item Generador de UHF (transceptor ALINCO UHF),
    \item Vatímetro de UHF,
    \item Fuente de alimentación $12V$,
    \item Línea de transmisión $Z_O = 50 \Omega$,
    \item Impedancias de carga de prueba $Z_{L1}$,
    \item Cables varios (numerados 10, 11, 12) para interconexión (ya instalados),
    \item Conectores BNC hembra-hembra, BNC Triple hembra (T) y adaptador UHF hembra a N macho.
  \end{itemize}

  \section{Actividades}
    \begin{enumerate}
      \item Conectar el generador a la frecuencia de trabajo $f = 410 MHz$.
      \item Con el vatímetro, verificar la potencia entregada por el transmisor (debe indicar LOW).
      \item Conectar la impedancia de carga $Z_L$ al vatímetro.
      \item Medir la potencia incidente $P_i$  y la potencia reflejada $P_r$.
        \begin{equation*}
          P_i = *W \quad P_r = *W
        \end{equation*}
      \item Calcular el coeficiente de reflexión $\Gamma$ y la Relación de Onda Estacionaria ROE sin ningún stub (ss) conectado a la línea de transmisión.
        \begin{equation*}
          \Gamma_{ss} = * \quad ROE_{ss} = *
        \end{equation*}
      \item Comprobar en base a lo calculado en el ejercicio que el Stub y la distancia al Stub sean iguales.
      \item Conectar el adaptador BNC tipo T luego del vatímetro.
      \item Conectar la carga mediante el segmento de línea calculado y en el extremo libre, conectar el stub adecuado.
      \item Medir la potencia incidente $P_i$  y la potencia reflejada $P_r$.
        \begin{equation*}
          P_i = *W \quad P_r = *W
        \end{equation*}
      \item Calcular el coeficiente de reflexión $\Gamma$ y la Relación de Onda Estacionaria ROE con el stub (cs) conectado a la línea de transmisión.
        \begin{equation*}
        \Gamma_{cs} = * \quad ROE_{cs} = *
        \end{equation*}
    \end{enumerate}

  \section{Cuestionario}
    \begin{enumerate}
      \item ¿Qué implica que la línea de transmisión esté adaptada?
      \item ¿Qué conclusión de obtiene con las mediciones de solamente la carga y después con la introducción de un Stub?
      \item ¿Qué función tiene la carga utilizada?
    \end{enumerate}

  \section{Resolución}
  
  \begin{enumerate}
  
    \item Primero configuramos el generador a la frecuencia de trabajo especificada por la consigna.
  
    \begin{figure}[H]
      \centering
      \includegraphics[width=0.6\textwidth]{pictures/setup_gen.jpeg}
      \caption{Generador UHF configurado en 410\,MHz.}
      \label{fig:gen}
    \end{figure}
  
    \item Luego de verificar que la potencia entregada por el generador es 0\,W, conectamos la carga $Z_1$.
  
    \begin{figure}[H]
      \centering
      \begin{minipage}[b]{0.48\textwidth}
        \centering
        \includegraphics[width=\textwidth]{pictures/vatimetro_en_0.jpeg}
        \caption{Lectura que confirma la ausencia de potencia en el vatímetro.}
        \label{fig:vatimetro}
      \end{minipage}
      \hfill
      \begin{minipage}[b]{0.48\textwidth}
        \centering
        \includegraphics[width=\textwidth]{pictures/Z1_conectada.jpeg}
        \caption{Carga $Z_1$ conectada.}
        \label{fig:z1}
      \end{minipage}
    \end{figure}
  
    \item Procedemos a medir potencia incidente y reflejada en la carga sin adaptación.
  
    \begin{figure}[H]
      \centering
      \includegraphics[width=0.6\textwidth]{pictures/Medicion sin adaptacion.jpeg}
      \caption{Setup de medición de potencia en la carga sin adaptación.}
      \label{fig:setup_sin_adapt}
    \end{figure}
  
    \begin{figure}[H]
      \centering
      \begin{minipage}[b]{0.48\textwidth}
        \centering
        \includegraphics[width=\textwidth]{pictures/Potencia incidente sin adaptacion.jpeg}
        \caption{Lectura de potencia incidente en la carga sin adaptación.}
        \label{fig:pi_sin}
      \end{minipage}
      \hfill
      \begin{minipage}[b]{0.48\textwidth}
        \centering
        \includegraphics[width=\textwidth]{pictures/potencia reflejada Sin adaptacion.jpeg}
        \caption{Lectura de potencia reflejada en la carga sin adaptación.}
        \label{fig:pr_sin}
      \end{minipage}
    \end{figure}
  
    En base a estas mediciones tenemos:
    \[
    P_i = 3.6\,\text{W} \quad\text{y}\quad P_r = 0.6\,\text{W}
    \]
  
    \item A partir de los datos anteriores calculamos el coeficiente de reflexión y la relación de onda estacionaria (ROE) sin \emph{stub} conectado a la línea:
  
    \[
    |\Gamma_{ss}| = \sqrt{\frac{P_r}{P_i}} = \sqrt{\frac{0.6}{3.6}} = 0.408
    \]
    \[
    ROE_{ss} = \frac{1 + |\Gamma_{ss}|}{1 - |\Gamma_{ss}|} = \frac{1.408}{0.592} = 2.38
    \]
  
    Por lo tanto:
    \[
    \boxed{|\Gamma_{ss}| = 0.41 \qquad ROE_{ss} = 2.38}
    \]
  
    \item Luego, con un conector BNC en T, adaptamos la línea de transmisión mediante un \emph{stub} en circuito abierto.
  
    \begin{figure}[H]
      \centering
      \includegraphics[width=0.6\textwidth]{pictures/Linea adaptada.jpeg}
      \caption{Línea adaptada con \emph{stub} en circuito abierto.}
      \label{fig:linea_adaptada}
    \end{figure}
  
    \begin{figure}[H]
      \centering
      \begin{minipage}[b]{0.48\textwidth}
        \centering
        \includegraphics[width=\textwidth]{pictures/potencia incidente con adaptacion.jpeg}
        \caption{Lectura de potencia incidente en la carga con adaptación.}
        \label{fig:pi_con}
      \end{minipage}
      \hfill
      \begin{minipage}[b]{0.48\textwidth}
        \centering
        \includegraphics[width=\textwidth]{pictures/potencia reflejada con adaptacion.jpeg}
        \caption{Lectura de potencia reflejada en la carga con adaptación.}
        \label{fig:pr_con}
      \end{minipage}
    \end{figure}
  
    En base a estas mediciones tenemos:
    \[
    P_i = 4\,\text{W} \quad\text{y}\quad P_r = 0\,\text{W}
    \]
  
    El coeficiente de reflexión resulta:
    \[
    |\Gamma_{ss}| = \sqrt{\frac{P_r}{P_i}} = \sqrt{\frac{0}{4}} = 0
    \]
  
    Luego, la Relación de Onda Estacionaria (ROE) es:
    \[
    ROE_{ss} = \frac{1 + |\Gamma_{ss}|}{1 - |\Gamma_{ss}|} = 1
    \]
  
    Por lo tanto:
    \[
    \boxed{|\Gamma_{ss}| = 0 \qquad ROE_{ss} = 1}
    \]
  
    \noindent
    Esto indica una adaptación perfecta de la línea, sin potencia reflejada.
  
  \end{enumerate}

  \section{Respuestas al cuestionario}
  
  \begin{enumerate}
    \item \textbf{¿Qué implica que la línea de transmisión esté adaptada?}
  
    Implica que la impedancia característica de la línea es igual a la impedancia de carga, es decir, $Z_L = Z_0$.  
    En esa condición no se produce reflexión de potencia, el coeficiente de reflexión es nulo $(|\Gamma|=0)$ y toda la energía entregada por el generador se transfiere a la carga.  
    Como consecuencia, la Relación de Onda Estacionaria (ROE) es igual a 1, indicando una transmisión ideal sin ondas estacionarias.
  
    \item \textbf{¿Qué conclusión se obtiene con las mediciones de solamente la carga y después con la introducción de un \emph{stub}?}
  
    Al medir únicamente con la carga, se observa cierta potencia reflejada, lo que evidencia desadaptación entre la línea y la carga.  
    Luego, al conectar el \emph{stub} en la posición adecuada, éste compensa la reactancia presente y corrige la desadaptación, reduciendo el coeficiente de reflexión a cero.  
    De esta forma, el sistema alcanza una condición de adaptación perfecta, donde la potencia reflejada desaparece y la ROE se reduce a 1.
  
    \item \textbf{¿Qué función tiene la carga utilizada?}
  
    La carga representa la impedancia que debe absorber la potencia proveniente del generador a través de la línea.  
    Su función es transformar la energía entregada en calor o trabajo útil, dependiendo del tipo de carga, y permitir el análisis de cómo las reflexiones y las adaptaciones afectan la transferencia de potencia.  
    En este experimento, se usa principalmente como elemento de referencia para evaluar la eficiencia de la adaptación mediante el \emph{stub}.
  \end{enumerate}
