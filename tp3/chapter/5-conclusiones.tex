\chapter{Conclusiones}

El desarrollo del presente trabajo permitió comprobar experimentalmente los principios de adaptación de líneas de
transmisión mediante un \emph{stub simple}.  A partir de las mediciones realizadas, se observó que sin el stub la línea
presentaba una Relación de Onda Estacionaria (ROE) de 2.38, lo que evidenció una desadaptación entre la carga y la
línea, con una potencia reflejada significativa.  Tras la incorporación del stub, los valores obtenidos fueron $P_i =
4\,\mathrm{W}$ y $P_r = 0\,\mathrm{W}$, logrando una ROE igual a 1 y confirmando la adaptación perfecta.

Estos resultados demuestran la eficacia del método de adaptación mediante stub, ya que permitió eliminar la componente
reactiva de la impedancia total y maximizar la transferencia de potencia hacia la carga.  Asimismo, se verificó la
correspondencia entre los cálculos teóricos y las mediciones prácticas, validando el uso del Ábaco de Smith como
herramienta de diseño para determinar la posición y longitud óptimas del stub.

En conclusión, la experiencia permitió comprender de manera integral la relación entre reflexión, adaptación e
impedancias en líneas de transmisión, consolidando los conceptos fundamentales de propagación de ondas y transferencia
eficiente de energía en sistemas de radiofrecuencia.
