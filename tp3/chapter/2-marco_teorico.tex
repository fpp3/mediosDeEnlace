\chapter{Marco Teórico: Adaptación con un Stub Simple}

La teoría que sustenta este trabajo se basa en la propagación de ondas electromagnéticas y los principios de
transferencia de energía en líneas de transmisión.

\section{Fundamentos de la propagación y la reflexión}

El flujo de potencia a lo largo de una línea se describe mediante el \textbf{Vector de Poynting}, que representa la
densidad de energía transportada por la onda electromagnética.  Cuando la onda incidente encuentra una discontinuidad
—una carga con impedancia $Z_L$ distinta de la impedancia característica $Z_0$—, una parte de la energía se refleja
hacia el generador.  La superposición de las ondas incidente y reflejada da origen a una \textbf{onda estacionaria},
cuyo análisis permite evaluar la eficiencia de la transmisión.

\section{Cuantificación de la desadaptación}

La desadaptación se cuantifica mediante dos parámetros esenciales:

\begin{itemize} \item \textbf{Coeficiente de reflexión ($\Gamma$)}: mide la relación entre la onda reflejada y la
  incidente. \[ \Gamma = \frac{Z_L - Z_0}{Z_L + Z_0} \quad\Rightarrow\quad |\Gamma| = \sqrt{\frac{P_r}{P_i}} \] \item
  \textbf{Relación de Onda Estacionaria (ROE)}: expresa el grado de desadaptación y se calcula como: \[ ROE = \frac{1 +
    |\Gamma|}{1 - |\Gamma|} \] Una ROE igual a 1 indica una línea perfectamente adaptada, sin potencia reflejada.
\end{itemize}

\section{Método de adaptación con \emph{stub}}

La adaptación busca eliminar las reflexiones y lograr que la impedancia total vista por el generador sea igual a $Z_0$.
Para ello se emplea un \emph{stub}, o ramal sintonizado, que consiste en un tramo corto de línea de transmisión
conectado en paralelo a la línea principal y terminado en circuito abierto o cortocircuito.  

La admitancia de un \emph{stub} es puramente imaginaria, lo que permite usarlo para cancelar la componente reactiva de
la impedancia de carga.  El procedimiento consiste en identificar un punto de la línea donde la admitancia normalizada
tenga parte real igual a 1 y parte imaginaria $jB_{residual}$.  El \emph{stub} se dimensiona con una susceptancia
$jB_{stub} = -jB_{residual}$, logrando así una admitancia total unitaria ($1 + j0$) y, por ende, adaptación perfecta.

\section{Aplicación práctica y uso del Ábaco de Smith}

El \textbf{Ábaco de Smith} se utiliza como herramienta gráfica para representar impedancias y admitancias a lo largo de
la línea.  Permite determinar visualmente la distancia desde la carga hasta el punto donde debe colocarse el \emph{stub}
(d) y su longitud física ($L_s$), expresadas en longitudes de onda o en centímetros.  

En el laboratorio, el procedimiento consiste en: \begin{enumerate} \item Medir las potencias incidente ($P_i$) y
  reflejada ($P_r$) sin el \emph{stub}, para obtener la ROE inicial. \item Calcular los parámetros de adaptación ($d$ y
  $L_s$) mediante el análisis teórico y el Ábaco de Smith. \item Conectar el \emph{stub} y repetir las mediciones. \item
  Comparar la nueva ROE con la inicial, verificando la reducción de la potencia reflejada. \end{enumerate}

  Cuando el sistema se encuentra correctamente adaptado, la ROE tiende a 1 y el coeficiente de reflexión a 0,
  confirmando la máxima transferencia de potencia.
